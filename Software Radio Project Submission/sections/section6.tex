In Part V of the project we must account for a time-varying timing phase which was inserted in the transmitted signal. In this case the optimal timing phase may not be every L samples from the initial transmitted symbol. This can arise due to non-ideal crystal oscillators and we therefore must determine at which decimation points we should grab the transmitted signal to extract the payload. In this part we must do an initial coarse timing recovery based on the theory discussed on Section II on the ensemble average power. Once that is done we find the maximum power peak and treat that as the beginning of decimation sequence on our transmitted packed. From this transmitted packet we use the previously discussed cross-correlation peak analysis, shown in Figure~\ref{fig:cross-corr-parti}. This allows us to detect the beginning of our payload. Once the payload has been detected we utilize a decision-directed timing recovery algorithm. The decision-directed method uses basic differential calculus to minimize the expectation of the error signal, $\xi = E[\abs{{e[n]}}]^2$ where $e[n] = s[n] - y(nT_b + \tau)$. Taking the derivative of the error to find a local minimum solution depending on our initialization point gives (unless function is convex) gives $\dfrac{d\epsilon}{d\tau} =  -2R\{(e^{\star}[n]\dfrac{dy(nT_b + \tau)}{d\tau})\}$. This gradient descent method then utilizes an approximation of this derivative to slowly move towards the optimal phase in each iteration.

Utilizing this algorithm on xRF9.mat shows the need to timing phase restoration. Without timing phase restoration, the QPSK constellation is of the form in Figure~\ref{fig:no-dd}. When timing phase and slicing is applied to latch to the closest symbol in the desired constellation our QPSK constellation is of the form in Figure~\ref{fig:with-dd} with the correct text output.
In a receiver three functions operate concurrently: timing recovery, carrier recovery, and channel equalization. If the channel equalizer tap-weight adaptation algorithm is fast enough one may automatically compensate for an residual carrier offset $\Delta f_c$. Any phase offset can be accounted for with the use of a carrier recovery loop at the output of the equalizer, as well as the tap weights rotate to help compensate for any phase offset. This leads to a receiver system shown in Figure~\ref{fig:phase-pll}. As mentioned earlier, a fractionally-spaced equalizer can perform timing recovery, therefore the timing and carrier recovery blocks can be removed just leaving the PLL and equalizer loops. The current implementation of the LMS algorithm has slow convergence as shown in Figure~\ref{}. The use of a faster adapting algorithm is needed for carrier offset tracking, such as the regularized least squares algorithm (RLS). The RLS algorithm projects the under-determined system to the solution with the smallest error vector within the systems vector space. The convergence characteristics shown in Figure~\ref{} are an order of magnitude better then the LMS algorithm.
Part III of the project requires us to implement a half-symbol spaced equalizer shown in Figure~\ref{fig:frac-equal}. Fractionally-spaced equalizers are more complex than symbol spaced since they operate at a higher data rate then the symbol spaced equalizers, however hold benefits that warrant the complexity increase. One key benefit that fractionally-spaced equalizers bring is that if they learning rate is fast enough they can deal with timing phase imperfections without the need for an explicit timing recovery method. The core theory behind this benefit is that timing phase imperfections are brought about by incorrect phase in spectral aliasing. When the data rate is high enough to avoid the spectral aliasing the timing phase will not depend on the phase of the overlap anymore as long as the fractionally-spaced equalizer is used correctly.

To correctly train the tap-weights we once again utilize the cyclic method shown in Figure~\ref{fig:cyclic-frac}. It is important to note that because of the expansion by L we have to grab L*32 samples of the preamble to get all symbols in the pilot. In the case of the half-symbol spaced equalizer L = 2 and M=1 where M is the decimation factor.To ensure the correct tap weights are associated with pilot symbols we must shift the preamble sequence by 2. Since the pilot is still 32 symbols we must perform this cyclic equalization over a two clock cycle span for each iteration. The general conclusions of the symbol-spaced equalizers are applicable to the fractionally spaced equalizer as well.

The eye pattern of the half-symbol spaced equalizer is shown for xRF3.mat in Figure~\ref{fig:payload-equal-part3}. As shown, the constellation becomes closer to ideal. Mathematically we can enumerate the performance of the equalizers by looking at the mean-square error of the two signals. 
Part I loads the transmission data from an ideal channel. We may assume there is zero phase error and zero frequency offset. The transmitter performs modulation with a carrier wave of the form $cos(2\pi f_c t)$ where $f_c = 10^5 Hz$. This is determined by looking at the spectral content of the signal \textit{xRF1.mat} shown in Figure~\ref{fig:xbb_partI}, which mimics the shape of a square-root raised cosine frequency response around $f_c = 10^5 Hz$ as expected. To increase the received signal to have highest SNR ratio we utilize matched filtering. This match filter also operates as a low-pass filter which can remove all unwanted spectral content outside the baseband. The transmitter utilizes expansion to reduce the bandwidth of the transmitted signal, therefore we needed to decimate the signal at the correct timing phase to ensure the signal samples are correctly identified at the receiver. According to chapter 10, for pulses with less then 100\% excess bandwidth 3 spectral coefficients exist in the Fourier series of the ensemble power $E[y(t)^2]$ of the received signal. This leads to the closed-form solution for the ensemble average $E[y(\tau)^2] = {\rho}_0 + 2\abs{{\rho}_1} cos(\dfrac{2\pi}{T_b}\tau + \angle{{\rho}_1})$ where ${\rho}_0$ is the  component and ${\rho}_1$ is the first harmonic. The ensemble average is a dc-shifted sinusoid, Figure~\ref{fig:ensemble-power}, which was verified through the Matlab code. Since there is not channel distortion the optimal timing phase is the location of the peak of the sinusoidal graph. After some time to reach steady state, the peaks occur every 100 samples as expected due to expansion by 100. We find the first sample that correlates to the peaks which is sample 1 and then decimate by every 100 samples from there. To verify the preamble and payload are correctly deciphered we may look at the constellation shown in Figure~\ref{fig:parti-packet-constellation}. As shown we see that the data symbols are at the correct symbol mapping spots with the preamble being scattered around these four points.

It is stated that the preamble begins consists of four pilot sequences each of length 32. Therefore, if we cross-correlate the decimated baseband signal with the pilot sequence we obtain four peaks, as shown in Figure~\ref{fig:cross-corr-parti}. These peaks indicated strong correlation between the baseband signal and the pilot, which means that the two sequences are the same. Therefore if we know the pilot is N=32 symbols then 32 symbols after the last peak the payload begins. Since there is no channel distortion we should expect a perfect QPSK constellation as shown in Figure~\ref{fig:payload-constellation-parti}. Passing this data to the text file we are able to correctly decipher xRF1.mat which tells a joke.


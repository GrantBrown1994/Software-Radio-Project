In Part II we explore how channel distortion can cause problems with ISI and how the use of equalizers can be used to solve non-ideal channel variance. To explore the mathematical model of the channels impulse response we note that the channel can be modeled as $c(t) = {\alpha}_i \delta (t-{\tau}_i)$ assuming a single-path channel. When using Nyquist pulses the equivalent baseband impulse response between the transmitter and receiver is $c_{BB}(t) = {\beta}_i p(t-{\tau}_i)$, where $p(t)$ is the convolution between the matched receiver and transmitter filters, and ${\beta}_i={\alpha}_i e^{-j2\pi f_c {\tau}_i}$ is a complex gain factor. If we assume that the delay incurred by the channel is insignificant the received signal at the demodulator gives $y(t) = \sum_{n=-\infty}^{\infty} s[n]c_{BB}(t -nT_b) + v(t)$ where $v(t)$ is noise added by a non-ideal channel. If we include the complex gain factor ${\beta}_i$ the complex baseband equivalent impulse response of the channel simplifies to $y(t) = {\beta}_i \sum_{n=-\infty}^{\infty} s[n]c_{BB}(t -nT_b) + v(t)$. In the absence of channel noise we have perfect Nyquist pulses spaced by $T_b$ at the output and sampling at these portions we can recover the transmitted symbols. When there is a significant delay in the channel or the channel has additive noise our assumptions fail and we must revert to the more general impulse response of $y(t) = \sum_{n=-\infty}^{\infty} s[n]c_{BB}(t -nT_b) + v(t)$. Here $c_{BB}(t)$ is not a perfect Nyquist pulse and ISI will occur. To restore the Nyquist pulse after the receiver we pass $y(t)$ through an equalizer, $w(t)$ whose cascade formation with the baseband channel impulse response, $c_{BB}(t) \star w(t)$, restores the Nyquist pulse shape to an acceptable level of ISI. The addition of multi-path channels follows the above discussion with an emphasis on the delay of the paths must be equal to ensure no ISI, otherwise an equalizer is needed.

In Part II we implement a symbol-spaced equalizer, as shown in Figure~\ref{fig:symbol-spaced}. By transmitting a sequence of pilot symbols we can utilize the cyclic equalization method shown in Figure~\ref{fig:symbol-cyclic}. The goal is to minimize the mean squared error between the equalized symbols and the pilot, $\abs{{e[i]}^2}$. To correctly utilize the cyclic equalizer we need to grab the preamble from the received signal. This can be done by performing an auto-correlation of the received and decimated signal. Theory says that when the transmitted signal is auto-correlated through the equation $r_{yy}(n, N+1) = \sum_{k=0}^{N} y[n-k]y^{\star}[n-N-1-k]$ we will obtain a flat top or constant auto-correlation, as shown in Figure~\ref{fig:auto-corr}, when transversing over the repeated pilot sequence. Grabbing any 32 symbol sequence from this flat section results in single period of the pilot sequence which is then used in the cyclic equalizer. The code to implement the cyclic equalizer is shown in Figure~\ref{fig:symbol-code}. The output of the weights may be shifted, as shown in Figure~\ref{fig:non-circ-weights}, due to an offset of where you grabbed the preamble and the at which symbol you start the pilot sequences. Since equalizer tap weights peak in the middle of the sequence and the channel impulse response also peaks in the middle we circularly shift the tap weights to be center aligned as shown in Figure~\ref{fig:circ-weights}. The effect of the equalizer can be shown through the a comparison of the constellation of the payload before and after equalization in Figure~\ref{fig:payload-equal-part2}.

Correct timing phase is critical for performance because the symbol spaced equalizer does not correct timing phase errors. Figure~\ref{fig:payload-equal-wrong-phase-part2} shows that the constellation of the decimated signal after equalization becomes less clear when compared to the optimal timing phase decimated sequence due to increased ISI. We will defer more detailed timing phase discussion until Part IV.
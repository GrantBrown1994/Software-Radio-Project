As demonstrated in Figure~\ref{fig:comm-system}, transmission of signals in digital systems requires the use of: a transmitter, a channel, a receiver, and a equalizer. The job of the transmitter is to generate data symbols which are free of ISI that can be transmitted over the desired channel medium. This is accomplished by first encoding bits to data symbols, which are then passed to a Nyquist pulse shaper, and then modulated to the desired carrier frequency. The receiver must then demodulate the desired signal to the baseband, perform matched filtering to remove unwanted spectrum's, and then sample at the correct timing phase to obtain the transmitted symbols. Physical devices, such as crystal oscillators are used in this process which bring physical limits to the precision of demodulation and sampling precision. These limits bring arise two major issues. One issue is carrier offset, which occurs because the transmitted signal is frequency shifted by a carrier $cos(2\pi f_c + \Theta (t))$. To perform exact demodulation we must know the timing varying phase as well as the exact frequency at which the signal was modulated. Any inaccuracies can cause a slight frequency offset where out signal is not centered around 0 Hz and can result in a constellation rotation shown in Figure~\ref{fig:qpsk-rotate}. The other issue is timing recovery, which occurs because exactly recovering the signal after demodulation and filtering requires knowing the exact timings at which the Nyquist pulse was generated. These timings are referred to as the optimal timing phase. Any delta between the transmission timing phase and receiving timing phase can introduce ISI and disturb the Nyquist pulses.
In addition, desired channel mediums typically will introduce some level of distortion to the received signals spectrum through the addition of unwanted noise and auxiliary signals. It is imperative to remove this noise to restore the Nyquist pulse properties to ensure no ISI and correct symbol detection. Equalizers are designed to perform this deconvolution operation which restores the initial spectrum by correctly weighting the irregularities through trainable weights, similar to a Neural Network. 

The project explores these issues and potential solutions and implementations of these solutions through a 6-sectioned project. Most of the solutions introduced used in this project are pilot-aided or decision-directed methods. The transmitted signals consist of a preamble and payload as shown in Figure~\ref{fig:data_sequence}. The preamble consists of training symbols, which are called pilots, which are used to fine tune the receiver parameters to ensure the correct signals are received at the output of the subsystem. Part I focuses on the basics of recovering the timing phase from an ideal channel. Part II and III introduces channel irregulaitires into the design and requires symbol-spaced and fractionally-spaced equalizers to be implemented to restore the Nyquist pulses which removes ISI. Part IV explores carrier and phase offset and the rotation it causes in constellations. Part V requires students to utilize decision-directed methods to restore a drifting optimal timing phase. Part VI is optional and uses fractionally-spaced equalizers to perform all functions of the receiver.


%Part I implements the demodulation and timing recovery sequences in the receiver with a ideal channel.  We must determine where the preamble and payload sequences are located in the transmitted data to ensure correct detection of the signals. Part II explores channel distortion and allows us to implement a symbol-spaced equalizer to correctly undo the effects of the channel. Part III investigates the fractionally-spaced equalizer and how that can be utilized to undo the effects of the channel. Part IV investigates the carrier offset and phase offset situations that occur in communication systems and implement data-aided methods to correctly undo the frequency and phase offset. Part V introduces timing phase drift into the transmitted signals and requires us to use data-aided timing phase recovery mechanisms to correctly decode the transmitted signal. Part VI explores the use of fractionally spaced equalizers to perform phase, frequency, and timing offset of carriers.
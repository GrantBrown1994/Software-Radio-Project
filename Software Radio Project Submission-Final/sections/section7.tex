In a receiver three functions operate concurrently: timing recovery, carrier recovery, and channel equalization. If the channel equalizer tap-weight adaptation algorithm is fast enough one may automatically compensate for an residual carrier offset $\Delta f_c$. However, if the carrier offset it too large then the channel equalizer cannot resolve this issue, therefore a coarse carrier recovery is needed at the beginning. As mentioned earlier, a fractionally-spaced equalizer can perform timing recovery, therefore the timing and carrier recovery blocks can be removed. This leads to a receiver system shown in Figure~\ref{fig:phase-pll} without the PLL, since the equalizer can solve for the phase rotation. The current implementation of the LMS algorithm has a convergence rate that is too low to track carrier offset issues with the constellation. The use of a faster adapting algorithm is needed for carrier offset tracking, such as the recursive least squares algorithm (RLS). The RLS algorithm projects the under-determined system to the solution with the smallest error vector within the systems vector space. The convergence characteristics shown in Figure~\ref{fig:lms-conv} and Figure~\ref{fig:rls-conv} show that the RLS convergence is an order of magnitude better then the LMS algorithm.
Part IV explores having carrier and phase offset in the receiver and channel and how that affects the constellations. The carrier $cos(2\pi f_c t + \Theta (t))$ is modulated by the data symbols at the transmitter. $\Theta (t)$ is a time varying phase that must be accounted for to avoid constellation rotation shown in Figure~\ref{fig:phase-rotate}. In addition to a time-varying phase the receiver must perform demodulation to move the transmitted data symbols back to baseband. If the carrier frequency and the demodulation sinusoid frequency are not exactly equal we will obtain resulting in the signal $y(t) = e^{j2\pi \Delta f_c t} \sum_{n=-\infty}^{\infty} s[n] h_0 (t-nT_b) + v(t)$ where $h_0(t)$ is the impulse response of the channel to the first data symbol. To obtain the ideal baseband signal of $y(t) = \sum_{n=-\infty}^{\infty} s[n] h_0 (t-nT_b) + v(t)$ we must determine the carrier offset $\Delta f_c$. Utilizing the pilot we can utilize the auto-correlation of $J = \sum_{n=N_1}^{N_2} y[n+N] y^{\star}[n]$. Expanding this signal we know $x[n] = x[n+N]$ which will give us $J=e^{j2\pi \Delta f_c t}\sum_{n=N_1}^{N_2}\abs{{x[n]}^2}$, which holds due to the uncorrelated nature of the noise $v(t)$ and the transmitted symbols $x[n]$. Solving for $\Delta f_c$ we get: $\Delta f_c = \dfrac{1}{2\pi NT_b}\angle{J}$. Note that the length of the periodic signal N determines the precision to what we can determine in carrier offset.

After the equalizer we may still be dealing with some residual phase offset of the signals. To alleviate this problem before detecting the final data we pass it through a data-directed phase tracking method shown in Figure~\ref{fig:phase-pll}. The theory behind this method is a predictive phase detector, which compares the slicer input and output. If the phase error is small enough then we can approximate the error as $\epsilon [n] = \dfrac{I\{\~s[n] \^s[n]\}}{R\{\~s[n] \^s[n]\}}$ where $\~s[n] \text{and} \^s[n]$ are the slicer inputs and outputs. This error is then passed to a tune-able loop filter which updates the phase for the next symbol.

Utilizing this procedure on xRF7.mat we can see the effect of the carrier phase offset in the constellation in Figure~\ref{fig:before-dd}. Applying the decision directed method to equalize the phase we get the constellation in Figure~\ref{fig:after-dd}, showing the need for the carrier phase offset removal.